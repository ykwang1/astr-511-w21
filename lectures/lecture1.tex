\documentclass[letterpaper,landscape]{slides}
\usepackage{boxedminipage}

\input pdf.tex

\setlength{\topmargin}{-1in}
\setlength{\textheight}{7.5in}
\setlength{\textwidth}{9in}
\setlength{\oddsidemargin}{0pt}
\setlength{\oddsidemargin}{0pt}


\begin{document}
\newcommand{\XXX}[1]{\textbf{XXX} #1}
\newcommand{\colour}[1]{\color{#1}}

\def\eq#1{\begin{equation} \color{blue} #1 \end{equation}}
\def\b#1{{\bf  #1}}
\def\p{\partial}
\def\th{^{th}}
\def\msun{{\rm\,M_\odot}}
\def\bnabla{{\bf\nabla}}
\def\dint{\int\!\!\!\int}
\def\d{{\rm d}}
\def\i{{\rm i}}
\def\ddt#1{{\rm{d} #1\over {\rm dt}}}
\def\ddtS#1{{\rm{d^2} #1\over {\rm dt^2}}}
\def\spose#1{\hbox to 0pt{#1\hss}}
\def\lta{\mathrel{\spose{\lower 3pt\hbox{$\mathchar"218$}}
     \raise 2.0pt\hbox{$\mathchar"13C$}}}
\def\gta{\mathrel{\spose{\lower 3pt\hbox{$\mathchar"218$}}
     \raise 2.0pt\hbox{$\mathchar"13E$}}}
\def\mspace{\hbox{\quad}}
\def\deffn#1{{\bf#1}}\def\eqs#1{equations \rf#1}


\newcount\itemCnt\itemCnt=0
\newcommand{\nitem}{%
  \global\advance\itemCnt by 1
  ~\vskip0cm\the\itemCnt.\qquad}

\definecolor{orange}{rgb}{1.0, 0.5, 0.0}
\definecolor{purple}{cmyk}{0.4, 0.8, 0.3, 0.0}

%%%%%%%%%%%%%%%%%%%%%%%%%%%%%%%%%%%
\newcommand{\picslide}[7]{%
  \begin{slide}
     \begin{center}
        \begin{minipage}{#5in}
            \hskip #6in
            \hskip -1in
            {\scalebox{#4}{\includegraphics{figures/#1.#2}}}
            \vskip #7in
            {\large \color{blue} #3}
        \end{minipage}
     \end{center}
     \vfill
  \end{slide}
}
%%%%%%%%%%%%%%%%%%%%%%%%%%%%%%%%%%%

%%%%%%%%%%%%%%%%%%%%%%%%%%%%%%%%%%%
\newcommand{\QAslide}[1]{%
  \begin{slide}
     \begin{center}
        \begin{minipage}{7in}
            \phantom{x} \vskip -0.7in
            \phantom{x} \hskip  0.5in
            {\scalebox{0.8} {\includegraphics{figures/#1.pdf}}}           
        \end{minipage}
     \end{center}
 \end{slide}
}
%%%%%%%%%%%%%%%%%%%%%%%%%%%%%%%%%%%

\newcommand{\smgraphicsZI}[7]{
   \begin{minipage}[t]{#6in}
     \vskip#3in \hskip#4in
     \scalebox{#5}{\includegraphics{figures/#1.#2}}
   \end{minipage}}


\newcommand{\palV}[2]{
\begin{slide}
\begin{minipage}{8in}
~\vskip-1in
\rotatebox{0}{\scalebox{0.85}{\includegraphics{figures/#1}}}
\vskip -2.5in~
\end{minipage}

#2

\vfill
\end{slide}
}

%------------------------------------------------------------------------------
%------------------------------------------------------------------------------
%------------------------------------------------------------------------------

\begin{slide}

\phantom{x}
\vskip -2in
\begin{center}
\bfseries
{\large {\color{blue} Astr 511: Galaxies as galaxies}}
\end{center}

{\centerline {\color{blue}  University of Washington}}
{\centerline {\color{blue} Mario Juri\'{c} \& \v{Z}eljko Ivezi\'{c}}}
\vskip 1in

{\centerline {\huge {\color{red}      Lecture 1:             }}}
\vskip 0.2in 
{\centerline {\Large {\color{blue} Review of Stellar Astrophysics,    }}}
{\centerline {\Large {\color{blue}   Star Clusters, and more...       }}}

\vfill
\end{slide}

%------------------------------------------------------------------------------

 
%------------------------------------------------------------------------------
%
%\begin{slide}
%\begin{center}
%{\large \color{red} Understanding the Properties of Galaxies and the Milky Way  }
%\end{center}
%
%
%{Binney \& Tremaine: \color{blue} ``Always majestic, often spectacularly 
%beautiful, galaxies are the fundamental building blocks of the Universe.''}
%
%
%{\bf The goals of this class are:}
%
%\begin{itemize}
%\item
%Understanding the correlations between various galaxy 
%properties using simple physical principles; discussion
%of the formation and evolution of galaxies
%\item 
%Understanding in detail the Milky Way structure (distribution
%of stars and ISM, stellar kinematics, metallicity and age
%distributions)  
%\item
%Reproducing some published work
%\end{itemize}  
%
%\vfill
%\end{slide}
%------------------------------------------------------------------------------

%------------------------------------------------------------------------------
% TWO-SIDED PAGE 
%\begin{slide}
%
%\hbox to \hsize{
%\begin{minipage}[t]{9cm}
%\begin{center}
%\vskip -0.85in
%\scalebox{0.72}{\hskip -1.5in \includegraphics{figures/HubbleSDSS.jpg}}
%\end{center}
%\end{minipage}
%
%\begin{minipage}[t]{15cm}
%\begin{center}
%\vskip -1in
%{\large \color{red} Learning about Galaxies }
%\end{center}
%
%\begin{itemize}
%\item
%Galaxies are (mostly) made of stars (also gas, dust, active galactic nuclei -- AGN); 
%hence have similar (but not identical!) color distributions
%\item
%They come in various shapes and forms (spiral vs. ellipticals; aka exponential
%vs. de Vaucouleurs profiles)
%\item
%Some host active nuclei (AGNs), some have high star-formation rates, some are very unusual
%(dwarf galaxies, mergers, etc.)
%\item
%We are typically interested in various distribution functions (e.g. for 
%luminosity, colors, mass, age, metallicity, size, etc.) -- the hope is to
%figure out {\color{blue} how galaxies formed and evolved}
%\item
%Nearest neighbors: the Andromeda galaxy (M31), Large and Small Magellanic Clouds, 
%the Sgr Dwarf (may be more)
%\end{itemize}  
%
%\end{minipage}}
%\vfill 
%\end{slide}
%------------------------------------------------------------------------------

\begin{slide}
	\begin{center}
		{\large \color{red} 
			Outline
		}
	\end{center}
	
	\begin{enumerate}
		\item {\color{blue} Stars:} a 5-minute review of stellar physics
		\item {\color{blue} Stellar parameters:} (mass, age, chemical composition) vs. 
		(temperature, surface gravity, metalicity)
		\item {\color{blue} Open and Globular clusters:} simple stellar populations 
		\item {\color{blue} Population Synthesis:} cooking up a galaxy
		\item {\color{blue} What do we observe:} a summary of the measurement process
		\item {\color{blue} Virial Theorem:} very brief intro to a very useful tool
	\end{enumerate}          
	\vfill
\end{slide}




%------------------------------------------------------------------------------


\begin{slide}
\begin{center}
{\large \color{red} 
                                Flash Review: the Basics
}
\end{center}


{\color{blue} I believe you're generally familiar with these terms (at an undergraduate level):}

\begin{itemize}
\item {\color{blue}general:} absolute magnitude, distance modulus, 
                             bolometric luminosity, the Planck function
\item {\color{blue}types of stars:} main sequence, white dwarfs, horizontal branch, 
                           red giants, supergiants, subgiants, subdwarfs, etc.
\item {\color{blue}stellar properties:} effective temperature, spectral class, 
                           metallicity, mass, age, MK (Morgan–Keenan) spectral classification
\end{itemize}     
     
\vfill
\end{slide}




%------------------------------------------------------------------------------
% TWO-SIDED PAGE 
\begin{slide}

\hbox to \hsize{
\begin{minipage}[t]{12cm}
\begin{center}
\vskip -0.6in
\scalebox{0.45}{\hskip -1.0in \includegraphics{figures/HR.jpg}}
\vskip 0.1in
\scalebox{0.4}{\hskip  -8.5in \includegraphics{figures/2forces.jpg}}
\vskip -1.9in
\scalebox{0.25}{\hskip  8.5in \includegraphics{figures/pic29691.jpg}}
\vskip -1.1in
\scalebox{0.25}{\hskip 22.5in \includegraphics{figures/pic35878.jpg}}
\vskip -0.4in
\end{center}
\vskip -0.5in

\end{minipage}

\begin{minipage}[t]{12cm}
\begin{center}
\phantom{xxx}
\vskip -1in
{\large \color{red} Hertzsprung-Russell Diagram }
\end{center}


\begin{itemize}
\item {\color{blue} Stars are balls of hot gas} in hydrodynamical and thermodynamical equilibrium
\item {\color{blue} Equilibrium based on two forces}, gravity: inward, radiation pressure: outward
\item {\color{blue} Temperature and size} cannot take arbitrary values: the allowed ones are 
     summarized in HR diagram 
\item {\color{red} $L = {\rm Area} \, {\rm  x} \, {\rm  Flux} = 4\pi R^2 \sigma T^4$}
\item {\color{blue} Luminosity and size} span a {\bf huge} dynamic range!
\end{itemize}

\end{minipage}}
\vfill 
\end{slide}
%--------------------------------------------------------------------------------------------


%------------------------------------------------------------------------------
% TWO-SIDED PAGE 
\begin{slide}

\hbox to \hsize{
\begin{minipage}[t]{12cm}
\begin{center}
\vskip -0.8in
\scalebox{0.75}{\hskip -1.4in \includegraphics{figures/HR_ages.jpg}}
%\vskip -0.007in
\scalebox{0.55}{\hskip -1.8in \includegraphics{figures/Smolcic_fig8a.jpg}}

\end{center}
\end{minipage}

\begin{minipage}[t]{12cm}
\begin{center}
\vskip -1in
{\large \color{red} HR Diagram: Stellar Age }
\end{center}


\begin{itemize}
\item {\color{blue} The main sequence} is where most of lifetime is spent.
\item {\color{blue} The position on the main sequence} is determined by mass! 
\item {\color{blue} The lifetime depends on mass:} massive (hot and blue) stars
                    have {\bf much} shorter lifetimes than red stars
\item {\color{blue} After a burst of star formation,} blue stars disappear 
                    {\bf very quickly}, $10^8$ years or so
\item {\color{blue} Galaxies are made of stars:} if there is no ongoing 
         star formation, they are red; if blue, there {\bf must} be actively
         making stars!
\item Turn-off color depends on both age and metallicity (later...) 
\end{itemize}

\end{minipage}}
\vfill 
\end{slide}
%--------------------------------------------------------------------------------------------


%------------------------------------------------------------------------------
% TWO-SIDED PAGE 
\begin{slide}

\hbox to \hsize{
\begin{minipage}[t]{12cm}
\begin{center}
\vskip -2.8in
\scalebox{1.1}{\hskip -2.5in \includegraphics{figures/GaiaHRdiagram_Babusiaux_fig8.pdf}}
%\vskip -0.007in
%\scalebox{0.55}{\hskip -1.8in \includegraphics{figures/Smolcic_fig8a.jpg}}

\end{center}
\end{minipage}

\begin{minipage}[t]{12cm}
\begin{center}
\vskip -1in
{\large \color{red} Gaia Early Data Release 3! }
\end{center}


\begin{itemize}
\item EDR3 released in Dec 2020
\item  {\color{blue}  1,811,709,771 sources} 
\item Pan-STARRS and Gaia catalogs now contain more than a billion sources each! 
\item Unprecedented combination of sky coverage, depth and astrometric accuracy (trigonometric parallax and proper motion measurements) 
\item Figure from Babusiaux et al. (2018, A\&A 616, A10): a breath-taking HR diagram for field stars! 
\end{itemize}

\end{minipage}}
\vfill 
\end{slide}
%--------------------------------------------------------------------------------------------




\begin{slide}
	\begin{center}
		\begin{minipage}{6in}
			\vskip -0.3in
			\hskip -1.8in
			{\scalebox{0.9}{\includegraphics{figures/Gaia_Eyer2019_fig2.pdf}}}
			\vskip -0.1in
			{Eyer et al. (2019, A\&A 623, A110)}    
		\end{minipage}
	\end{center}
	\vfill
\end{slide}

 

%------------------------------------------------------------------------------
% TWO-SIDED PAGE 
\begin{slide}

\hbox to \hsize{
\begin{minipage}[t]{12cm}
\begin{center}
\vskip -0.8in
\scalebox{0.55}{\hskip -1.4in \includegraphics{figures/Kurucz_Bstars.jpg}}
%\vskip -0.007in
\scalebox{0.55}{\hskip -1.4in \includegraphics{figures/Kurucz_Gstars.jpg}}

\end{center}
\end{minipage}

\begin{minipage}[t]{12cm}
\begin{center}
\vskip -1in
{\large \color{red} Stellar Parameters }
\vskip -1in
\end{center}


\begin{itemize}
\item {\color{blue} The stellar spectral energy distribution} is a function of 
      {\bf mass}, {\bf chemical composition} and {\bf age}, a theorist would say
\item {\color{blue} The stellar spectral energy distribution} is a function of 
      {\bf effective temperature}, {\bf surface gravity} and {\bf metallicity}; an observer would observe. E.g., {\color{blue} Kurucz models (1979)} 
      describe SEDs of (not too cold) main sequence stars, as a function of $T_{\rm eff}$, 
      $log(g)$ and $[Fe/H]$ at the accuracy level of 1\%.
\item These two parametrizations are connected by {\color{blue} stellar evolution models.}
\end{itemize}

\end{minipage}}
\vfill 
\end{slide}
%--------------------------------------------------------------------------------------------





%------------------------------------------------------------------------------
% TWO-SIDED PAGE 
\begin{slide}

\hbox to \hsize{
\begin{minipage}[t]{9cm}
\begin{center}
\vskip -0.6in
\scalebox{0.45}{\hskip -1.7in \includegraphics{figures/NGC2419-core.jpg}}
\vskip 0.1in
\scalebox{0.30}{\hskip -2.1in \includegraphics{figures/NGC2419-gr.jpg}}

\end{center}
\end{minipage}

\begin{minipage}[t]{15cm}
\begin{center}
\vskip -1in
{\large \color{red} Open and Globular Clusters}
\end{center}


\begin{itemize}
\item {\bf Top left:} SDSS gri composite image of globular cluster NGC 2419;
         note blue (literally) horizontal branch stars and yellowish (red)
         giants; the image is color-coded by the observed  g-r
\item {\bf Bottom left:} the SDSS g vs. g-r color-magnitude diagram of the area 
       around NGC 2419; the dots are
       color-coded by the observed SDSS g-r color
       % note the close correspondence with the top panel!
\item {\bf Below:} the SDSS gri composite image of open cluster M67 (NGC 2682)
\end{itemize}
\vskip 0.0in
\scalebox{0.29}{\hskip 1.7in \includegraphics{figures/NGC2682.jpg}}

\end{minipage}}
\vfill 
\end{slide}
%--------------------------------------------------------------------------------------------


%------------------------------------------------------------------------------
% TWO-SIDED PAGE 
\begin{slide}

\hbox to \hsize{
\begin{minipage}[t]{12cm}
\begin{center}
\vskip -0.8in
\scalebox{0.75}{\hskip -1.4in \includegraphics{figures/HR_ages.jpg}}
%\vskip -0.007in
\scalebox{0.55}{\hskip -1.8in \includegraphics{figures/Smolcic_fig8a.jpg}}

\end{center}
\end{minipage}

\begin{minipage}[t]{12cm}
\begin{center}
\vskip -1in
{\large \color{red} Open and Globular Clusters}
\end{center}


\begin{itemize}
\item Stellar clusters are excellent probes of stellar astrophysics
\item {\color{blue} Some key properties:} 
   \begin{enumerate}
      \item All stars at roughly the same distance
      \item All stars have roughly the same composition
      \item All stars have roughly the same age
   \end{enumerate}
\item The position of the main sequence, at a given color, 
       depends on metallicity 
\item The turn-off color depends on age and metallicity 
\item Other features, such as morphology of blue horizontal 
   branch and red giant branch, also depend on age and metallicity 

\end{itemize}

\end{minipage}}
\vfill 
\end{slide}
%--------------------------------------------------------------------------------------------




\begin{slide}
	\begin{center}
		\begin{minipage}{6in}
			\vskip -0.5in
			\hskip -2.2in
			{\scalebox{0.9}{\includegraphics{figures/GaiaHRdiagram_Babusiaux_fig3.pdf}}}
			\vskip -0.5in
			{Gaia EDR3 data for 14 GCs vs [Fe/H]: \,\,\,\,\,\,\,\,\,\,\,\,\,\,\,\,\,\, Babusiaux et al. (2018, A\&A 616, A10)}
		\end{minipage}
	\end{center}
	\vfill
\end{slide}

 
%--------------------------------------------------------------------------------------------



%------------------------------------------------------------------------------
% TWO-SIDED PAGE 
\begin{slide}

\hbox to \hsize{
\begin{minipage}[t]{11cm}
\begin{center}
\vskip -0.6in
\scalebox{1.0}{\hskip -0.7in \includegraphics{figures/ClusterLumFn.jpg}}
\vskip 0.4in
\scalebox{0.5}{\hskip -1.7in \includegraphics{figures/Pal13.jpg}}
\end{center}
\end{minipage}

\begin{minipage}[t]{13cm}
\begin{center}
\vskip -1in
{\large \color{red} Open and Globular Clusters}
\end{center}


\begin{itemize}
\item Open clusters are younger and concentrated towards the Galactic plane;
      globular clusters are more spherically distributed, at larger distances
      from the galactic center, have much lower metallicity and are much older
\item {\bf Top left:} luminosity functions for globular and open clusters 
       are very different 
\item {\bf Bottom left:} globular cluster Pal 13 (Siegel et al. 2001, 
      AJ 121, 935): Spatial profiles of globular clusters usually closely 
      follow {\bf King profiles} (c.f. later lecture on stellar dynamics);
      next slide stolen from Doug Heggie
\end{itemize}

\end{minipage}}
\vfill 
\end{slide}
%--------------------------------------------------------------------------------------------

\picslide{heggie1}{jpg}{UV-radio surveys}{0.92}{7}{-0.9}{3.5}

%------------------------------------------------------------------------------
% TWO-SIDED PAGE 
\begin{slide}

\hbox to \hsize{
\begin{minipage}[t]{11cm}
\begin{center}
\vskip -0.3in
\scalebox{1.05}{\hskip -0.9in \includegraphics{figures/Pal5tailsRaw.jpg}}
%\vskip -0.007in
\vskip 0.3in
\scalebox{1.05}{\hskip -0.9in \includegraphics{figures/Pal5tails.jpg}}

\end{center}
\end{minipage}

\begin{minipage}[t]{13cm}
\begin{center}
\vskip -1in
{\large \color{red} Tidal Tails around Globular Clusters }
\end{center}


\begin{itemize}
\item {\bf Top left:} SDSS stellar counts around globular clusters 
             M5 and Pal5 
\item {\bf Bottom left:} matched filter extraction of tidal tails around
         Pal 5 (gray: SFD E(B-V)) by Rockosi et al. (2002) and
         Odenkirchen et al. (2003)
\item For more details about matched filter method, see Grillmair (2008,
       arXiv:0811.3965; and references therein)
\item {\color{blue} Tidal tails provide strong constraints on the Milky 
       Way gravitational potential.} 
\item {Overall distribution constrains the mass 
	   (e.g., Eadie \& Juri\'{c} 2019, ApJ, 875, 159).} 
\end{itemize}

\end{minipage}}
\vfill 
\end{slide}
%-----------------------------------------------------------------------------



\picslide{Pal5orbit}{jpg}{}{0.82}{7}{-0.9}{3.5}


	
\picslide{prfig4}{jpg}{}{0.82}{7}{0.0}{3.5}




%------------------------------------------------------------------------------
% TWO-SIDED PAGE 
\begin{slide}
	
	\hbox to \hsize{
		\begin{minipage}[t]{6cm}
			\begin{center}
				
				\vskip 1.1in
				\scalebox{0.7}{\hskip -1.3in \includegraphics{figures/mw.jpg}}
				
				
			\end{center}
		\end{minipage}
		
		\begin{minipage}[t]{17cm}
			\begin{center}
				\vskip -1in
				{\large \color{red} Properties of GC Population}
			\end{center}
			
			
			\begin{itemize}
				\item
				{\color{blue} Halo GCs claim to fame: Shapley used their distribution to
					demonstrate that the Sun is not in the center of the Milky Way} 
			\end{itemize}
			
			\vskip 1in
			\scalebox{0.7}{\hskip 1.3in \includegraphics{figures/gc_galaxy.jpg}} 
			
	\end{minipage}}
	\vfill 
\end{slide}
%--------------------------------------------------------------------------------------------



%------------------------------------------------------------------------------
% TWO-SIDED PAGE 
\begin{slide}
	
	\hbox to \hsize{
		\begin{minipage}[t]{6cm}
			\begin{center}
				The age vs. metallicity distribution of globular clusters from
				Percival \& Solaris. 
				
				\vskip 0.3in
				\scalebox{0.5}{\hskip -2.3in \includegraphics{figures/GC_FeH.jpg}}
				
				
			\end{center}
		\end{minipage}
		
		\begin{minipage}[t]{17cm}
			\begin{center}
				\vskip -1in
				{\large \color{red} Properties of GC Population}
			\end{center}
			
			
			\begin{itemize}
				\item
				Halo GCs claim to fame: Shapley used their distribution to
				demonstrate that the Sun is not in the center of the Milky Way
				\item {\color{blue} Most globular clusters are metal-poor}, and thus resemble halo stars. 
				Their spatial distribution is also halo-like: roughly spherically distributed,
				and at distances of tens of kpc from the galactic center. Kinematics are
				similar to halo stars: randomly oriented eccentric orbits. 
				\item 
				{\color{blue} About 20\% of GCs have higher metallicities ($-1 < [Fe/H] < 0$)} and are
				found within 1-2 kpc from the galactic plane. Their distribution and 
				kinematics are very similar to thick disk.
				\item
				These differences are probably due to processes that happened early in the
				history of the Milky Way. It is likely that ``thick disk clusters'' formed
				after halo clusters. 
			\end{itemize}
			
	\end{minipage}}
	\vfill 
\end{slide}
%--------------------------------------------------------------------------------------------


%------------------------------------------------------------------------------
% TWO-SIDED PAGE 
%\begin{slide}
%	
%	\hbox to \hsize{
%		\begin{minipage}[t]{12cm}
%			\begin{center}
%				
%				\vskip 0.3in
%				\scalebox{0.9}{\hskip -1.1in \includegraphics{figures/gc_hr.jpg}}
%				
%				
%			\end{center}
%		\end{minipage}
%		
%		\begin{minipage}[t]{11cm}
%			\begin{center}
%				\vskip -1in
%				{\large \color{red} Properties of GC Population}
%			\end{center}
%			
%			
%			\begin{itemize}
%				\item
%				Standard modern compilation of globular cluster properties (n=157): 
%				Harris, W.E. 1996, {\it A Catalog of Parameters for Globular Clusters in the Milky Way},
%				Astronomical Journal, 112, 1487  (updated in 2010) 
%				\item Note the existance of the so-called {\it blue straggler} population in the H-R
%				diagram (left): it shouldn't exist given the cluster turn-off age; it could be the result 
%				of stellar collisions and mergers 
%			\end{itemize}
%			
%	\end{minipage}}
%	\vfill 
%\end{slide}
%--------------------------------------------------------------------------------------------

%------------------------------------------------------------------------------
% TWO-SIDED PAGE 
\begin{slide}
	
	\hbox to \hsize{
		\begin{minipage}[t]{6cm}
			\begin{center}
				
				\vskip 0.1in

				\scalebox{0.7}{\hskip -1.0in \includegraphics{figures/mw.jpg}}
				
				
			\end{center}
		\end{minipage}
		
		\begin{minipage}[t]{20cm}
			\begin{center}
				\vskip 1in
				{\large \color{red} From Clusters to Field Stars:}
			\end{center}
			
			\begin{center}
             	\vskip 1in
            	{\large \color{black} Calibrating Distances}
            \end{center}
			
%			\begin{itemize}
%				\item
%				{\color{blue} Halo GCs claim to fame: Shapley used their distribution to
%					demonstrate that the Sun is not in the center of the Milky Way} 
%			\end{itemize}
%			
%			\vskip 1in
%			\scalebox{0.7}{\hskip 1.3in \includegraphics{figures/gc_galaxy.jpg}} 
			
	\end{minipage}}
	\vfill 
\end{slide}
%--------------------------------------------------------------------------------------------



%------------------------------------------------------------------------------
% TWO-SIDED PAGE 
\begin{slide}

\hbox to \hsize{
\begin{minipage}[t]{8cm}
\begin{center}
\vskip -0.8in
\scalebox{0.5}{\hskip -1.6in \includegraphics{figures/M5_cmd1.jpg}}
\vskip -0.2in
\scalebox{0.5}{\hskip -2.1in \includegraphics{figures/GCs_cmd1.jpg}}

\end{center}
\end{minipage}

\begin{minipage}[t]{16cm}
\begin{center}
\vskip -1in
{\large \color{red} Photometric Parallax Calibration}
\end{center}


\begin{itemize}
\item {\color{blue} With a sufficiently large sample of globular clusters, we can calibrate 
      $M_r(g-i,[Fe/H])$, and then apply it to field stars to get their distances.}
\item {\bf Top Left:} an example of a globular cluster (M5) as observed by
       SDSS; the line is a polynomial fit to the median main sequence (large 
       circles):
\begin{equation}
       M_r = M_r^{0.6} -2.85 + 6.29 \, (g-i) -2.30 \, (g-i)^2  \nonumber
\end{equation}
where $M_r^{0.6}$ is the median absolute magnitude for stars 
with $0.5<g-i<0.7$. 
\item {\bf Bottom Left:} this is a good fit to a number of globular 
  clusters observed by SDSS, showing that the main effect of metallicity
  is to slide the main sequence vertically (i.e. along luminosity axis,
  changing $M_r^{0.6}$), without much effect on its {\it shape}.
\end{itemize}

\end{minipage}}
\vfill 
\end{slide}
%--------------------------------------------------------------------------------------------


%------------------------------------------------------------------------------
% TWO-SIDED PAGE 
\begin{slide}

\hbox to \hsize{
\begin{minipage}[t]{8cm}
\begin{center}
\vskip -0.8in
\scalebox{0.5}{\hskip -1.8in \includegraphics{figures/GCs_FeHshift.jpg}}
\vskip 0.1in
\scalebox{0.5}{\hskip -2.1in \includegraphics{figures/GCs_age.jpg}}

\end{center}
\end{minipage}

\begin{minipage}[t]{16cm}
\begin{center}
\vskip -1in
{\large \color{red} Photometric Parallax Calibration}
\end{center}


\begin{itemize}
\item {\color{blue} The position of the main sequence depends on [Fe/H]}
\item {\bf Top Left:} calibration (short-dashed) based on SDSS data (dots) and VandenBerg 
       \& Cleam (2003; squares). The shift is huge: $>$1 mag between median 
       halo metallicity ($[Fe/H]=-1.5$) and local disk metallicity  
       ($[Fe/H]=-0.2$).  {\color{blue} Must know [Fe/H] to within 0.2 dex
       to know distances within 10\%!}
\item {\bf Bottom Left:} at a {\bf fixed} [Fe/H], the turn-off depends on 
      the age of a stellar population (based on models!). 
\item For a relation appropriate for age of 10 Gyr (at halo metallicity) see
      eq. A7 in Ivezi\'{c} et al. (2008, ApJ, 684, 287)
\item {\it What about metallicity?}
\end{itemize}

\end{minipage}}
\vfill 
\end{slide}
%--------------------------------------------------------------------------------------------


%------------------------------------------------------------------------------
% TWO-SIDED PAGE 
\begin{slide}

\hbox to \hsize{
\begin{minipage}[t]{8cm}
\begin{center}
\vskip -0.8in
\scalebox{0.5}{\hskip -1.8in \includegraphics{figures/FeHspecUGR.jpg}}
\vskip 0.2in
\scalebox{0.58}{\hskip -1.9in \includegraphics{figures/FeHphotoVSspec.jpg}}

\end{center}
\end{minipage}

\begin{minipage}[t]{16cm}
\begin{center}
\vskip -1in
{\large \color{red} Photometric Metallicity Calibration}
\end{center}

\begin{itemize}
\item {\color{blue} At a fixed effective temperature, the amount of 
       UV flux ($\lambda<4000 \AA$) for F/G main-sequence stars is very 
       sensitive to metallicity (Wallerstein 1962)}
\item {\bf Top Left:} the dependence of spectroscopic metallicity 
      (using 60,000 SDSS stellar spectra) on the position in the $g-r$ vs. 
       $u-g$ color-color diagram 
\item {\bf Bottom Left:} the correlation between photometric metallicity,
      estimated using a two-dimensional third-order polynomial fit to the
      map shown in the top left panel, and spectroscopic metallicity; 
      the individual values agree with an rms of 0.26 dex (includes errors
      from both methods)
\item {\color{blue} For stars with $0.2<g-i<0.8$, [Fe/H] can be estimated 
      if the $u$ band photometry (or $U$ in Johnson system) is available.}
\end{itemize}

\end{minipage}}
\vfill 
\end{slide}
%--------------------------------------------------------------------------------------------



%------------------------------------------------------------------------------
% TWO-SIDED PAGE 
%\begin{slide}
%
%\hbox to \hsize{
%\begin{minipage}[t]{0cm}
%\begin{center}
%%\vskip 0.2in
%%\scalebox{0.58}{\hskip -1.9in \includegraphics{FeHphotoVSspec.jpg}}
%
%\end{center}
%\end{minipage}
%
%\begin{minipage}[t]{24cm}
%\begin{center}
%\vskip -1in
%{\large \color{red} How good are stellar models?}
%\end{center}
%
%\begin{itemize}
%\item {\color{blue} Good at predicting absolute magnitudes of main-sequence
%    stars to within 0.1-0.2 mag for stars with $T_{eff}>4000$ K.} 
%\item An excellent model database: Percival et al. (2009, ApJ, 690, 427)
%\item {\bf Bottom:} fig. 5 from An et al. (2009): lines are YREC+MARCS
%           models for age of 12.6 Gyr; bulls-eye symbol marks the Sun 
%\end{itemize}
%\vskip 0.1in
%\scalebox{0.8}{\hskip 0.1in \includegraphics{figures/An09_fig6.jpg}}
%
%\end{minipage}}
%\vfill 
%\end{slide}
%--------------------------------------------------------------------------------------------


%------------------------------------------------------------------------------
% TWO-SIDED PAGE 
%\begin{slide}
%
%\hbox to \hsize{
%\begin{minipage}[t]{12cm}
%\begin{center}
%\vskip -0.8in
%\scalebox{1.07}{\hskip -0.8in \includegraphics{figures/An09_fig28.jpg}}
%%\vskip 0.2in
%%\scalebox{0.58}{\hskip -1.9in \includegraphics{FeHphotoVSspec.jpg}}
%
%\end{center}
%\end{minipage}
%
%\begin{minipage}[t]{12cm}
%\begin{center}
%\vskip -1in
%{\large \color{red} The turn-off color}
%\end{center}
%
%\begin{itemize}
%\item {\color{blue} At a fixed metallicity, the turn-off color depends on
%       age} (the  color-age translation can be obtained from models).
%\item {\bf Left:} the dependence of turn-off color on metallicity and   
%       age for YREC+MARCS models (fig. 28 from An et al. 2009)
%\item For example, at [Fe/H]=0, the turn-off color changes from 
%      $g-i=-0.2$ to $g-i=0.6$ as the age increases from 1 Gyr to 10 Gyr
%\item {\color{red} For age$\sim$10 Gyr, the change of color with age
%      is very small for all [Fe/H].} The gradient is about 0.02 mag/Gyr:
%      requires {\bf exquisite photometry!}  
%\end{itemize}
%
%\end{minipage}}
%\vfill 
%\end{slide}
%--------------------------------------------------------------------------------------------



%------------------------------------------------------------------------------
% TWO-SIDED PAGE 
%\begin{slide}
%
%\hbox to \hsize{
%\begin{minipage}[t]{10cm}
%\begin{center}
%\vskip -0.8in
%\scalebox{0.65}{\hskip -1.4in \includegraphics{figures/Chaboyer1.jpg}}
%%\vskip -0.007in
%
%An example of systematic effect: (unknown) He abundance
%\scalebox{0.60}{\hskip -1.8in \includegraphics{figures/Chaboyer2.jpg}}
%
%\end{center}
%\end{minipage}
%
%\begin{minipage}[t]{13cm}
%\begin{center}
%\vskip -1in
%{\large \color{red} Other Age Determination Methods }
%\end{center}
%
%
%\begin{itemize}
%\item {\color{blue} Globular cluster absolute ages provide a strong 
%   constraint on the age of the Universe;} however, the 
%   simultaneous effects of metallicity, distance, ISM extinction, 
%   He abundance, and various model parameters such as mixing length 
%   (see Chaboyer et al.; astro-ph/9706128), make age estimates
%   uncertain by about 25\% (Jimenez 1998, PNAS 95, 13) 
%\item Can we determine age and metallicity using stars that are 
%   not on main sequence? Independently of ISM extinction
%   correction and distance scale? 
%\end{itemize}
%
%\end{minipage}}
%\vfill 
%\end{slide}
%--------------------------------------------------------------------------------------------


%------------------------------------------------------------------------------
% TWO-SIDED PAGE 
%\begin{slide}
%
%\hbox to \hsize{
%\begin{minipage}[t]{11cm}
%\begin{center}
%\vskip -0.8in
%\scalebox{0.55}{\hskip -1.7in \includegraphics{figures/Jimenez1.jpg}}
%%\vskip -0.007in
%
%A comparison of errors for different methods (Jimenez 1998):
%
%\scalebox{0.40}{\hskip -2.3in \includegraphics{figures/Jimenez2.jpg}}
%
%\end{center}
%\end{minipage}
%
%\begin{minipage}[t]{13cm}
%\begin{center}
%\vskip -1in
%{\large \color{red} Other Age Determination Methods }
%\end{center}
%
%
%\begin{itemize}
%\item Can we determine age and metallicity using stars that are 
%   not on main sequence? Independently of ISM extinction
%   correction and distance scale? 
%\item {\color{blue} There are several other methods:} 
%   \begin{enumerate}
%      \item $\Delta V$ method: measure the magnitude difference between
%             the Horizontal Branch and the subgiant branch (Iben \& Renzini 1984);
%             about 0.1 mag/Gyr (but also depends on FeH, about 0.2 mag/dex)
%      \item Morphology of the Horizontal Branch (Jimenez 1998)
%      \item Luminosity Function \\ (Jimenez 1998)
%   \end{enumerate}
%\end{itemize}
%
%\end{minipage}}
%\vfill 
%\end{slide}
%--------------------------------------------------------------------------------------------


%------------------------------------------------------------------------------
% TWO-SIDED PAGE 
\begin{slide}
	
	\hbox to \hsize{
		\begin{minipage}[t]{6cm}
			\begin{center}
				
				\vskip 0.1in
				
				\scalebox{0.7}{\hskip -1.0in \includegraphics{figures/mw.jpg}}
				
				
			\end{center}
		\end{minipage}
		
		\begin{minipage}[t]{20cm}
			\begin{center}
				\vskip 1in
				{\large \color{red} From Clusters to Galaxies:}
			\end{center}
			
			\begin{center}
				\vskip 1in
				{\large \color{black} Population Synthesis}
			\end{center}
			
			%			\begin{itemize}
			%				\item
			%				{\color{blue} Halo GCs claim to fame: Shapley used their distribution to
			%					demonstrate that the Sun is not in the center of the Milky Way} 
			%			\end{itemize}
			%			
			%			\vskip 1in
			%			\scalebox{0.7}{\hskip 1.3in \includegraphics{figures/gc_galaxy.jpg}} 
			
	\end{minipage}}
	\vfill 
\end{slide}
%--------------------------------------------------------------------------------------------



%------------------------------------------------------------------------------
% TWO-SIDED PAGE 
\begin{slide}
	
	\hbox to \hsize{
		\begin{minipage}[t]{12cm}
			\begin{center}
				\vskip -0.8in
				\scalebox{0.55}{\hskip -1.7in \includegraphics{figures/popSynth_JD.jpg}}
				%\vskip -0.007in
				\scalebox{0.75}{\hskip -1.4in \includegraphics{figures/BC-youngold-SED-mor99.jpg}}
				
			\end{center}
		\end{minipage}
		
		\begin{minipage}[t]{12cm}
			\begin{center}
				\vskip -1in
				{\large \color{red} Population Synthesis: modeling the SED of a stellar system }
				\vskip 0.25in
			\end{center}
			
			
			\begin{enumerate}
				\item {\color{blue} A burst of star formation:} a group of stars (arbitrarily large) 
				was formed some time ago: {\bf age}
				\item {\color{blue} The mass distribution} of these stars is given by a function called
				{\bf initial mass function, IMF}, roughly a power-law $n(M) \propto M^{-3}$
				\item {\color{blue} The stellar distribution in the HR diagram} is given by the adopted
				age and IMF.%; equivalently, can adopt a CMD for a globular or open cluster; assume
				%{\bf metallicity} and get a model (i.e. stellar SED, e.g. from Kurucz) for each 
				%star and add them up
				\item {\color{blue} Metallicity also has a small but measurable effect.}
				%\item {\color{blue} Multiple bursts of star formation may be required to explain an
				%      observed SED of a galaxy.}
				\item This makes up a {\color{blue} Simple Stellar Population (SSP)}
			\end{enumerate}
			
	\end{minipage}}
	\vfill 
\end{slide}
%--------------------------------------------------------------------------------------------


%------------------------------------------------------------------------------
% TWO-SIDED PAGE 
\begin{slide}
	
	\hbox to \hsize{
		\begin{minipage}[t]{13cm}
			\begin{center}
				\vskip -0.8in
				\scalebox{0.55}{\hskip -1.7in \includegraphics{figures/popSynth_JD.jpg}}
				\vskip 0.4in
				\scalebox{0.65}{\hskip -5.8in \includegraphics{figures/popSynth_burst.jpg}}
				\vskip -3.6in
				\scalebox{0.65}{\hskip  3.4in \includegraphics{figures/popSynth_const.jpg}}
				
			\end{center}
		\end{minipage}
		
		\begin{minipage}[t]{11cm}
			\begin{center}
				\vskip -1in
				{\large \color{red} Population Synthesis: modeling SEDs of galaxies }
			\end{center}
			
			
			\begin{enumerate}
				%\item {\color{blue} A burst of star formation:} {\bf age}
				%\item {\color{blue} The initial mass function, IMF}
				%\item {\color{blue} The stellar distribution in the HR diagram} and metallicity: {\bf add 
				%                    SEDs for all stars}, the result is
				\item Adding the SEDs of stars within an SSP gives us the SSD of an SSP.
				Globular clusters are a good approximation of a single stellar population (more later).
				\item {\color{blue} Star-formation history}, or the distribution of stellar ages,
				tells us how to combine such simple stellar populations to get the SED of a realistic
				galaxy
			\end{enumerate}
			
	\end{minipage}}
	\vfill 
\end{slide}
%----------------------------------------------------------------------------


\picslide{RvsBgalaxies_JDsummary}{jpg}{}{1.0}{7}{-0.5}{0.1}

%------------------------------------------------------------------------------



\picslide{aa15217-10-fig8}{eps}{Above: NGC 891 (Popescu et al. 2011)}{2.0}{7}{-0.0}{0.5}


%------------------------------------------------------------------------------
\begin{slide}
	\begin{center}
		\bfseries
		{\large {\color{red} What do we measure? \color{blue} Radiation Intensity.}}
	\end{center}
	\vskip 0.6in
	
	{\centerline {\Huge \bf {\color{blue} 
				$I_\nu (\lambda, \alpha, \delta, t, {\bf p})$ }}}
	
	\begin{itemize}
		
		\item $I_\nu$ - energy (or number of photons) / time / Hz/ solid angle 
		\item $\lambda$ - $\gamma$-ray to radio, depending on resolution:
		spectroscopy, narrow-band photometry, broad-band photometry
		\item $\alpha, \delta$ - direction (position on the sky); the resolution 
		around that direction splits sources into unresolved (point) and resolved;
		interferometry, adaptive optics,...
		\item $t$ - time variability of emission
		\item {\bf p} polarization
	\end{itemize}
\end{slide}

%------------------------------------------------------------------------------



%------------------------------------------------------------------------------
\begin{slide}
	\begin{center}
		\bfseries
		{\large {\color{red} Instruments measure various integrals of this quantity. Examples:}}
	\end{center}
	
	\vskip 0.6in
	
	{\bf Imaging (photometry):}
	\begin{equation}
	I_\nu^{band}(<\alpha>, <\delta>, <t>) = \int_{0}^{\infty} S(\lambda)d\lambda 
	\int_{0}^T dt \int_{\theta}d\Omega  \, I_\nu (\lambda, \alpha, \delta, t, {\bf p}) 
	\end{equation}
	
	{\color{red} SDSS: \color{blue} T = 54.1 sec, $\theta \sim$1.5 arcsec, filter width $\sim$1000 \AA}
	
	{\bf Spectroscopy:}
	\begin{equation}
	F_{\nu}^{object}(\lambda, <t>) = \int_{0}^{\infty} R(\lambda)d\lambda 
	\int_{0}^T dt \int_{A}d\Omega  \, I_\nu (\lambda, \alpha_0, \delta_0, t, {\bf p}) 
	\end{equation}
	
	{\color{red} SDSS: \color{blue} T = 45 min, A: 3 arcsec fibers ($\sim$6 kpc at the redshift of 0.1),
		R$\sim$2 \AA \, ($\sim$70 km/s)}
	
\end{slide}

%------------------------------------------------------------------------------


\begin{slide}
	\begin{center}
		\begin{minipage}{6in}
			\vskip 1.5in
			\hskip -1.8in
			{\scalebox{0.7}{\includegraphics{figures/The-Sloan-Digital-Sky-Survey-SDSS-bandpasses-compared-to-a-range-of-stellar-spectra.png}}}
			\vskip 0.5in
			{\large \color{blue} SDSS filters and stellar SEDs}
		\end{minipage}
	\end{center}
	\vfill
\end{slide}

%\picslide{The-Sloan-Digital-Sky-Survey-SDSS-bandpasses-compared-to-a-range-of-stellar-spectra}{png}{SDSS photometric bandpasses.}{0.9}{6}{-1.3}{0.5} 

%------------------------------------------------------------------------------


\picslide{SDSS2MASSfilters}{pdf}{SDSS and 2MASS photometric bandpasses}{1.2}{7}{-0.7}{0.0} 




%------------------------------------------------------------------------------
\begin{slide}
	\begin{center}
		\bfseries
		{\large {\color{red} Even with the same instrument, multiple measurements are made}}
	\end{center}
	\vskip 0.6in
	
	\begin{itemize}
		\item {\color{blue} Magnitudes:} $m=-2.5 log(I_\nu / I_0)$ 
		\item {\color{blue} Magnitudes:} there are five different types in SDSS! Aperture,
		fiber, psf, model and Petrosian magnitudes.   
		% \item {\color{blue} Radial Profiles:} all magnitudes are measured 
		%        using circularized brightness profiles extracted for a predefined 
		%        set of radii
		\item {\color{red} Do we really need all these magnitudes?} 
	\end{itemize}
	
\end{slide}

%------------------------------------------------------------------------------


\picslide{NGC660-detail}{jpg}{SDSS and 2MASS photometric bandpasses}{0.7}{7}{-0.7}{3.5} 

%------------------------------------------------------------------------------
\begin{slide}
	\begin{center}
		\bfseries
		{\large {\color{red} An example: SDSS photometry}}
	\end{center}
	\vskip 0.6in
	
	\begin{itemize}
		\item {\color{blue} Magnitudes:} we need different magnitudes because, depending
		on an object's brightness profile, they {\color{blue} capture different information} and
		have {\color{blue} different noise properties.}
		\item {\color{blue} Unresolved sources:} {\color{red} aperture magnitudes} work well, but 
		only for bright stars; for a given error, {\color{red} psf magnitudes} go 1-2 mags deeper; 
		{\color{red} fiber magnitudes} measure flux within 3 arcsec aperture, and thus estimate 
		the flux seen by spectroscopic fibers
		\item {\color{blue} Resolved sources:} psf magnitudes don't include the total 
		flux, actually none of the various magnitudes includes the total flux
		for resolved sources! {\color{red} Petrosian magnitudes} include {\bf the same} 
		fraction of flux, independent of galaxy's angular size, however, they are
		very noisy for faint galaxies; {\color{red} model magnitudes} have smaller noise for
		faint galaxies (especially if you are interested only in colors)
	\end{itemize}
	
\end{slide}

%------------------------------------------------------------------------------


%------------------------------------------------------------------------------
%\begin{slide}
%\begin{center}
%\bfseries
%{\large {\color{red} The count (uncalibrated flux) extraction}}
%\end{center}
%\vskip 0.6in
%
%\begin{itemize}
%\item {\color{blue} In the limit of a circular source, all fluxes (magnitudes) 
%can be computed as:}
%
%         $flux(type) \propto \int p(x) \, \Phi(x) \, 2\pi x \,dx$ 
%
%\item {\color{blue} $type$:} aperture, fiber, psf, Petrosian, model
%\item {\color{blue} $p(x)$:} circularized brightness profile
%\item {\color{blue} $\Phi(x)$:} type-dependent weight function
%\begin{itemize}
%   \item {\color{red} aperture:} $\Phi(x)$ = 1 for $x<7.4$ arcsec, 0 otherwise
%   \item {\color{red} fiber:} $\Phi(x)$ = 1 for $x<1.5$ arcsec, 0 otherwise
%   \item {\color{red} psf:} $\Phi(x)$ = psf(x) for $x<3$ arcsec, 0 otherwise, 
%               {\it photo} uses 2D integration (angle dependence)
%   \item {\color{red} Petrosian:} $\Phi(x)$ = 1 for $x<R$ arcsec, 0 otherwise,
%      $R$ depends on the measured galaxy profile: defined by the 
%      ratio of the local surface brightness to the mean surface
%      brightness within the same radius  
%   \item {\color{red} model:} $\Phi(x)$ from a best-fit (deV or exp) 3-parameter 
%     pre-computed profile (convolved with seeing); must be 2D integration
%\end{itemize}
%
%For {\color{red} signal-to-noise calculation}, see document {\bf An LSST document on astronomical 
%signal-to-noise calculation and flux extraction} linked to the class webpage. 
%
%More information about {\color{red} SDSS galaxy photometry} can be found in Strauss et al. (2002, AJ 124, 1810). 
%
%\end{itemize} 
%
%\end{slide}
% 
%------------------------------------------------------------------------------


%------------------------------------------------------------------------------
%\begin{slide}
%\begin{center}
%\bfseries
%{\large {\color{red} Calibrated flux and magnitudes}}
%\end{center}
%\vskip 0.6in
%
%\begin{itemize}
%\item 
%Given a specific flux of an object {\it at the top} of the atmosphere, 
%$F_\nu(\lambda)$, a broad-band photometric system measures the in-band flux
%\begin{equation}
%\label{Fb}
%           F_b = \int_0^\infty {F_\nu(\lambda) \phi_b(\lambda) d\lambda},
%\end{equation}
%where $\phi_b(\lambda)$ is the normalized system response for a given 
%band (e.g. for SDSS $b=ugriz$)
%\begin{equation}
%\label{PhiDef}
%\phi_b(\lambda) = {\lambda^{-1} S_b(\lambda) \over \int_0^\infty {\lambda^{-1} S_b(\lambda) d\lambda}}.
%\end{equation}
%\item
%The overall atmosphere + system throughput, $S_b(\lambda)$, is obtained from  
%\begin{equation}
%\label{SDef}
%         S_b(\lambda) = S^{atm}(\lambda) \times S_b^{sys}(\lambda). 
%\end{equation}
%\end{itemize} 
%
%\end{slide}
% 
%------------------------------------------------------------------------------

%------------------------------------------------------------------------------
% TWO-SIDED PAGE 
%\begin{slide}
%
%\hbox to \hsize{
%\begin{minipage}[t]{10cm}
%\begin{center}
%\vskip -1.8in
%\scalebox{1.0}{\hskip -1.7in \includegraphics{figures/LSSTthroughput.pdf}}
%\end{center}
%\end{minipage}
%
%\begin{minipage}[t]{8cm}
%\begin{center}
%{\large \color{red} LSST throughput}
%\vskip -1in
%\end{center}
%
%\end{minipage}
%
%}
%
%\vfill 
%\end{slide}
%--------------------------------------------------------------------------------------------


%------------------------------------------------------------------------------
% TWO-SIDED PAGE 
%\begin{slide}
%
%\hbox to \hsize{
%\begin{minipage}[t]{10cm}
%\begin{center}
%\vskip -1.8in
%\scalebox{1.0}{\hskip -1.7in \includegraphics{figures/LSSTatmo.pdf}}
%\end{center}
%\end{minipage}
%
%\begin{minipage}[t]{8cm}
%\begin{center}
%{\large \color{red} Atmosphere}
%\vskip -1in
%\end{center}
%
%\end{minipage}
%
%}
%
%\vfill 
%\end{slide}
%--------------------------------------------------------------------------------------------



%------------------------------------------------------------------------------
%\begin{slide}
%\begin{center}
%\bfseries
%{\large {\color{red} Calibrated flux and magnitudes}}
%\end{center}
%\vskip 0.6in
%
%\begin{itemize}
%\item
%Photometric measurements are fully described by $F_b$ and its corresponding 
%$\phi_b(\lambda)$. The relevant temporal, spatial and wavelength scales 
%on which $\phi_b(\lambda)$ is known determine photometric accuracy.
%Typically, it is assumed that $\phi_b(\lambda)$ ``defines'' a photometric
%system (e.g. Johnson, Str\"omgren, SDSS)
%\item
%Traditionally, the in-band flux is reported on a magnitude scale
%\begin{equation}
%\label{ABmag}
%                m_b = -2.5\log_{10}\left({F_b \over F_{AB} }\right).
%\end{equation}
%where $F_{AB} = 3631$ Jy (1 Jansky = 10$^{-26}$ W Hz$^{-1}$ m$^{-2}$ 
%= 10$^{-23}$ erg s$^{-1}$ Hz$^{-1}$ cm$^{-2}$) is the flux normalization 
%for AB magnitudes (Oke \& Gunn 1983). These magnitudes are also called 
%``flat'' because for a source with ``flat'' spectral energy distribution 
%(SED) $F_\nu(\lambda)= F_0$, $F_b = F_0$. 
%\item
%Note: it might be a bit confusing that $F_\nu(\lambda)$ is integrated over
%wavelength in eq.~\ref{Fb}, and yet the result, $F_b$, has the same
%units as $F_\nu(\lambda)$. This happens because the product 
%$\phi_b(\lambda) d\lambda$ is dimensionless, and eq.~\ref{Fb} 
%formally represents weighting of $F_\nu(\lambda)$ rather than 
%its area integral. Of course, this is a consequence of the definition
%of AB system\footnote{The fact that $F_\nu(\lambda)$ is multiplied by 
%$S_b(\lambda)/\lambda$ and then integrated over wavelength is a 
%consequence of the fact that CCDs are photon-counting devices. That is,
%the units for $F_b$ are {\bf not} arbitrary. For more details, see 
%Maiz Apell\'{a}niz 2006 (AJ 131, 1184).} 
%in terms of $F_\nu(\lambda)$.  
%\end{itemize} 
%
%\end{slide}
% 
%------------------------------------------------------------------------------

%\picslide{SDSS2MASSfilters}{pdf}{SDSS and 2MASS photometric bandpasses}{1.2}{7}{-0.7}{3.5} 



%------------------------------------------------------------------------------
% TWO-SIDED PAGE 
\begin{slide}
	
	\hbox to \hsize{
		\begin{minipage}[t]{13cm}
			\begin{center}
				\vskip -0.8in
				\scalebox{0.75}{\hskip -1.3in \includegraphics{figures/SDSSMASSccd.pdf}}
			\end{center}
		\end{minipage}
		
		
		\begin{minipage}[t]{11cm}
			\vskip -0.8in
			\begin{center}
				{\large \color{red} Sources in Color-Color Diagrams (example: SDSS-2MASS)}
				\vskip 0.5in
			\end{center}
			
			\begin{itemize} 
				\item
				Blue/red: blue and red stars; green/magenta: blue and red galaxies,
				Circles: quasars ($z<2.5$)
				\item
				Optical/IR colors allow an efficient star-quasar-galaxy separation
				\item
				8-band accurate and robust photometry excellent for finding
				objects with atypical SEDs (e.g. red AGNs, L/T dwarfs, binary
				stars)
			\end{itemize} 
			
	\end{minipage}}
	\vfill 
\end{slide}



%------------------------------------------------------------------------------
% TWO-SIDED PAGE 
\begin{slide}
	
	\hbox to \hsize{
		\begin{minipage}[t]{9cm}
			\begin{center}
				\vskip -0.2in
				\scalebox{0.8}{\hskip -1.3in \includegraphics{figures/Zw1.jpg}}
			\end{center}
		\end{minipage}
		
		\begin{minipage}[t]{13cm}
			\begin{center}
				\vskip -1in
				{\large \color{red}  The Virial Theorem }
			\end{center}
			
			\begin{itemize}
				\item In a system of N particles, gravitational forces tend to 
				pull the system together and the stellar velocities tend to 
				make it fly apart. It is possible to relate kinetic and potential
				energy of a system through the change of its moment of intertia 
				\item In a {\color{blue} steady-state system}, these tendencies
				are balanced, which is expressed quantitatively through
				the {\color{blue} the Virial Theorem}. 
				\item {\color{blue} A system that is not in balance will tend to move
					towards its virialized state.} 
			\end{itemize}     
			
	\end{minipage}}
	\vfill 
\end{slide}
%------------------------------------------------------------------------------




%------------------------------------------------------------------------------
\begin{slide}
	\begin{center}
		{\large \color{red} 
			The Scalar Virial Theorem}
	\end{center}
	
	The Virial Theorem will be discussed in detail later in this class.  For now, 
	all we need to know is the final result for the {\color{blue} Scalar} Virial Theorem:
	the {\it average} kinetic and potential energy must be in balance:
	\eq{
		E = K + \Phi = -K = {1\over 2} \Phi
	} 
	where $K = M<v^2>/2$ is the kinetic energy, $\Phi$ is the gravitational potential
	energy, and $E$ is the total energy (negative for a gravitationally bound system). 

        For a bound system, the ``potential well'' is ``half full'': $K$ brings the level up 
        from $\Phi$  to $\Phi/2$ (e.g. from $-$100 J to $-$50 J). 
	
       Recall that $\Phi$ for two masses is 
	\eq{
		 \Phi = - G {mM \over r}. 
	} 
	
	\vfill
\end{slide}








%------------------------------------------------------------------------------
%\begin{slide}
%	\begin{center}
%		\vskip 1.5in
%		\scalebox{5.1}{\hskip -0.1in \includegraphics{figures/are_we_virialized.jpg}}
%	\end{center}
%	
%	\vfill
%\end{slide}




%------------------------------------------------------------------------------
\begin{slide}
	\begin{center}
		{\large \color{red} 
			The Scalar Virial Theorem: Applications  }
	\end{center}
	
	\begin{itemize}
		\item 
		If a system collapses from infinity, half of the potential
		energy will end up in kinetic energy, and the other half
		will be disposed of! From the measurement of the circular
		velocity and the mass of Milky Way (which constrain 
		the kinetic energy), we conclude that during their formation,
		galaxies radiate away about $3\times10^{-7}$ of their
		rest-mass energy.
		\item
		For a virialized spherical system, $M = 2 R \sigma^2 / G$. 
		We can estimate total mass from the size and velocity dispersion.
		E.g. for a cluster with $\sigma$=12 km/s, and R=3 pc, 
		we get $M = 2\times 10^5$ M$_\odot$ (note that $G=233$ in 
		these units)
		\item 
		{\color{red} Think about this for the next time:} Evil aliens
		give a ''kick'' to our Moon that increases its kinetic energy
		by 10\%. What will happen with the radius of its orbit (smaller or larger)? 
	\end{itemize}     
	
	\vfill
\end{slide}





%------------------------------------------------------------------------------
\end{document}


